\section{Simulaciones}\label{sec:simulaciones}

\subsection{Parámetros de entrada}\label{subsec:parametros-de-entrada}

Para poder llevar a cabo las simulaciones, se toma en cuenta un conjunto de parámetros de entrada.
Los mismos se pueden categorizar entre fijos y variables, dependiendo de si su valor se mantiene
constante a lo largo de todas las ejecuciones de la simulación para un mismo conjunto de reglas, o no.

Comenzando con los parámetros fijos, se tiene los límites de la matriz ($border$),
en el que se indican los valores máximos y mínimos de las coordenadas de las celdas en dos o tres dimensiones.
Luego, se define la condición de vecindad a partir de dos parámetros:
la forma en la que se consideran las celdas vecinas a una celda en particular, según los modelos mencionados
en~\ref{subsec:definiciones-de-vecindad} ($condition$), y el alcance de la vecindad ($r$).
Con respecto a las reglas de juego, se cuenta con dos parámetros que determinan las posibles cantidades de
vecinos que debe tener una celda para que se mantenga viva ($shouldKeepAlive$),
y para que una celda muerta se convierta en viva ($shouldRevive$); ambos son listas de enteros.
Por último, se determinó el dominio inicial ($initialDomainProportion$), el
cual representa una proporción del área, si es un simulación en dos dimensiones, o del volumen,
si es en tres dimensiones, total de la matriz donde se generan las celdas vivas en el paso temporal 0.
Es decir, siendo $A$ el área o volumen total de la matriz, el dominio inicial es $A \times initialDomainProportion$.

Por otro lado, los parámetros que se varían a lo largo de las simulaciones son la cantidad de pasos temporales
máximos ($maxIter$), y la proporción de densidad de celdas vivas en el dominio inicial ($initialLiveCellsProportion$),
siendo este último un valor entre 0 y 1.