\section{Simulaciones}\label{sec:simulaciones}

\subsection{Parámetros de entrada}\label{subsec:parametros-de-entrada}

Para poder llevar a cabo las simulaciones, se toma en cuenta un conjunto de parámetros de entrada.
Los mismos se pueden categorizar entre fijos y variables, dependiendo de si su valor se mantiene
constante a lo largo de todas las ejecuciones de la simulación para un mismo conjunto de reglas, o no.

Comenzando con los parámetros fijos, se tiene los límites de la matriz ($border$),
en el que se indican los valores máximos y mínimos de las coordenadas de las celdas en dos o tres dimensiones.
Luego, se define la condición de vecindad a partir de dos parámetros:
la forma en la que se consideran las celdas vecinas a una celda en particular, según los modelos mencionados
en la sección ~\ref{subsec:definiciones-de-vecindad} ($condition$), y el alcance de la vecindad ($r$).
Con respecto a las reglas de juego, se cuenta con dos parámetros que determinan las posibles cantidades de
vecinos que debe tener una celda para que se mantenga viva ($shouldKeepAlive$),
y para que una celda muerta se convierta en viva ($shouldRevive$); ambos son listas de enteros.
Por último, se determinó el dominio inicial ($initialDomainProportion$), el
cual representa una proporción entre $0$ y $1$ del área, si es un simulación en dos dimensiones, o del volumen,
si es en tres dimensiones, total de la matriz. 
Es decir, siendo $A$ el área o volumen total de la matriz, el dominio inicial es $A \times initialDomainProportion$.
La misma se ubica en el centro de la matriz, y es donde se generan las celdas vivas en el paso temporal inicial.

Por otro lado, los parámetros que se varían a lo largo de las simulaciones son la cantidad de pasos temporales
máximos ($maxIter$), y la proporción de densidad de celdas vivas en el dominio inicial ($initialLiveCellsProportion$),
siendo este último un valor entre 0 y 1.


\subsection{Definición de observables}\label{subsec:observables-posibles}

A continuación, se definen el conjunto de observables, mencionados en la sección \ref{sec:resultados}.
Los mismos se utilizan para analizar los resultados obtenidos en el equilibrio de las simulaciones
realizadas, en función de la densidad de celdas vivas en el dominio inicial.

\subsubsection{Cantidad de celdas vivas}\label{subsubsec:cantidad-de-celdas-vivas}
Este observable calcula la cantidad de celdas vivas en la matriz en el paso temporal de equilibrio.
Se define como:
\begin{equation}
    \text{Q} = \sum_{i=1}^{x} \sum_{j=1}^{y} \sum_{k=1}^{z} \mathbbm{1}_{\text{viva}}
    \label{eq:cantidad-celdas-vivas}
\end{equation}
donde $Q$ es la cantidad de celdas vivas, $x$, $y$ y $z$ son las dimensiones de la matriz, y $\mathbbm{1}_{\text{viva}}$
es una función indicadora que toma el valor de 1 si la celda en la posición $(i, j, k)$ está viva, y 0 en caso contrario.

\subsubsection{Variación de la cantidad de celdas vivas}\label{subsubsec:pendiente-de-crecimiento-de-cantidad-de-celdas-vivas}
Este observable calcula la pendiente de cantidad de celdas vivas en la matriz, entre el paso temporal
inicial y el paso temporal de equilibrio.
Se define como:
\begin{equation}
    \text{P}_{\text{Q}} = \frac{Q_{\text{eq}} - Q_{0}}{t_{\text{eq}} - t_{0}}
    \label{eq:pendiente-crecimiento-celdas-vivas}
\end{equation}
donde $P_{Q}$ es la variación de la cantidad de celdas vivas, $Q_{eq}$ es la
cantidad de celdas vivas en el paso temporal de equilibrio, $Q_{0}$ es la cantidad de celdas
vivas en el paso temporal inicial, $t_{eq}$ es el paso temporal de equilibrio, y $t_{0}$ es el paso temporal inicial.

\subsubsection{Tiempo de convergencia al equilibrio}\label{subsubsec:tiempo-de-convergencia-al-equilibrio}
Este observable calcula la cantidad de pasos temporales que tarda la simulación en converger al equilibrio.
Se define como:
\begin{equation}
    \text{T} = t_{\text{eq}} - t_{0}
    \label{eq:tiempo-convergencia-equilibrio}
\end{equation}
donde $T$ es el tiempo de convergencia al equilibrio, $t_{eq}$ es el paso temporal de equilibrio,
y $t_{0}$ es el paso temporal inicial.

\subsubsection{Distancia de la celda viva más alejada al centro}\label{subsubsec:distancia-de-la-celda-viva-mas-alejada-al-centro}
Este observable calcula la distancia de la celda viva más alejada al centro de la matriz
en el paso temporal de equilibrio.
Se define como:
\begin{equation}
    \text{D} = \max_{i, j, k | (i, j, k) \in \text{viva}} \left( \sqrt{(i - x/2)^2 + (j - y/2)^2 + (k - z/2)^2} \right)
    \label{eq:distancia-celda-viva-mas-alejada-al-centro}
\end{equation}
donde $D$ es la distancia de la celda viva más alejada al centro, $x$, $y$ y $z$ son las dimensiones de la matriz,
y $i$, $j$ y $k$ son las coordenadas de la celda en la posición $(i, j, k)$.

\subsubsection{Rapidez de alejamiento de la celda viva más alejada al centro}\label{subsubsec:rapidez-de-alejamiento-de-la-celda-viva-mas-alejada-al-centro}
Este observable calcula la pendiente de la distancia de la celda viva más alejada al centro de la matriz,
entre el paso temporal inicial y el paso temporal de equilibrio.
Se define como:
\begin{equation}
    \text{P}_{\text{D}} = \frac{D_{\text{eq}} - D_{0}}{t_{\text{eq}} - t_{0}}
    \label{eq:rapidez-alejamiento-celda-viva-mas-alejada-al-centro}
\end{equation}
donde $P_{D}$ es la rapidez de alejamiento de la celda viva más alejada al centro, $D_{eq}$ es la
distancia de la celda viva más alejada al centro en el paso temporal de equilibrio, $D_{0}$ es la
distancia de la celda viva más alejada al centro en el paso temporal inicial, $t_{eq}$ es el paso
temporal de equilibrio, y $t_{0}$ es el paso temporal inicial.








