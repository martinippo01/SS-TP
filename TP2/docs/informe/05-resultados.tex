\section{Resultados}\label{sec:resultados}
A continuación, se presentan los casos de estudio que se han realizado para diferentes simulaciones del juego
de la vida en dos y tres dimensiones, y sus respectivos resultados.
Para cada caso, se detallan los parámetros de entrada fijos utilizados, seguido de una o más figuras en el que
se muestra la evolución del sistema en cada paso temporal en valores extremos, y finalmente, un análisis del
observable en función de la densidad de celdas vivas en el dominio inicial.

\subsection{Conway en 2D}\label{subsec:conway-en-2d}

Como primer modelo, se ha estudiado el reconocido juego de la vida de Conway en dos dimensiones.
Para ello, se fija los siguientes parámetros de entrada:

\begin{itemize}
    \item $border = (0, 0) \times (100, 100)$
    \item $condition = MOORE$
    \item $r = 1$
    \item $shouldKeepAlive = [2, 3]$
    \item $shouldRevive = [3]$
    \item $initialDomainProportion = 0.16$
\end{itemize}
Variando el parámetro $initialLiveCellsProportion$ entre 0.1 y 0.9, se ha analizado la cantidad de celdas vivas
a lo largo de los pasos temporales.
\begin{figure}[H]
    \centering
    \includegraphics[width=0.6\linewidth]{conway2d/size_i10}
    \caption{Cantidad de celdas vivas en el tiempo del sistema de Conway con $initialLiveCellsProportion = 0.1$}
    \label{fig:conway2d_i10}
\end{figure}
\begin{figure}[H]
    \centering
    \includegraphics[width=0.6\linewidth]{conway2d/size_i90}
    \caption{Cantidad de celdas vivas en el tiempo del sistema de Conway con $initialLiveCellsProportion = 0.9$}
    \label{fig:conway2d_i90}
\end{figure}

De igual forma, se ha analizado la distancia de la celda viva más lejana al centro de la matriz en función del
paso temporal.

\begin{figure}[H]
    \centering
    \includegraphics[width=0.6\linewidth]{conway2d/distance_i10}
    \caption{Distancia de la celda viva más lejana al centro en función del tiempo con $initialLiveCellsProportion = 0.1$}
    \label{fig:conway2d_d10}
\end{figure}
\begin{figure}[H]
    \centering
    \includegraphics[width=0.6\linewidth]{conway2d/distance_i90}
    \caption{Distancia de la celda viva más lejana al centro en función del tiempo con $initialLiveCellsProportion = 0.9$}
    \label{fig:conway2d_d90}
\end{figure}

Como se puede apreciar, el sistema alcanza el equilibrio antes del paso temporal 900, por lo que se hace el análisis
del observable en el paso temporal mencionado.
Para este modelo, en función de la densidad de celdas vivas en el dominio inicial, se ha observado la cantidad de celdas vivas,
la pendiente de crecimiento de la cantidad de celdas vivas
