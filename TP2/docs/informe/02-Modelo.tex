\section{Modelo}\label{sec:modelo}

\subsection{Autómatas celulares}\label{subsec:automatas-celulares}

Un automata celular modela un sistema dinámico el cual evoluciona en pasos discretos mediantes reglas o heurísticas,
las que determinan la transición entre dos estados.

El sistema se compone de una matriz de celdas que poseen un número finito de estados discretos y estos se irán actualizando
de manera sincrónica en cada paso temporal.

Las reglas se caracterizan por ser determinísticas y uniformes en tiempo y espacio. Es decir, las reglas para la evolución 
de una celda solamente depende de su vecindario, el cual será definido según la implementación y las reglas.

En una transición de estados, se define el alcance r, el cual determina el grupo de celdas cercanas que deben ser 
tenidas en cuenta para la regla de transición.

\subsection{Definiciones de vecindad}\label{subsec:definiciones-de-vecindad}
Como se ha mencionado, el alcance es un parámetro que describe a un sistema en particualar. Sin embargo, debe tambien
tenerse en cuenta bajo que definicion de vecindario se trabaja. Se han contemplado los siguientes.

En primer lugar, el vecindario \textbf{Von Neumann} considera como vecina aquellas celdas que se encuentran dentro del rango r
considerando la distancia Manhattan. De forma mas precisa, el conjunto de vecinas se define como:
\begin{itemize}
    \item Bidimensional: $N_{i,j}^{(vN)}:=\{(a,b) \in L\ |\ |a - i| + |b - j| \leq r\}$.
    \item Tridimensional: $N_{i,j, k}^{(vN)}:=\{(a,b,c) \in L\ |\ |a-i| + |b - j| + |c + k| \leq r\}$.
\end{itemize}


Por otro lado, el vecindario \textbf{Moore} considera que una celda es vecina de otra si todas las componentes de su posición se encuentran
a una distancia menor a r. De forma mas precisa el conjunto de celdas vecinas es $N_{i,j}^{(M)}:=\{(k,l) \in L\ |\ |k-i|\leq r\ and\ |l - j| \leq r \}$.
\begin{itemize}
    \item Bidimensional: $N_{i,j}^{(M)}:=\{(a,b) \in L\ |\ |a-i|\leq r\ and\ |b - j| \leq r \}$
    \item Tridimensional: $N_{i,j,k}^{(M)}:=\{(a, b, c) \in L\ |\ |a-i|\leq r\ and\ |b - j|  \leq r\ and\ |c - k|  \leq r\}$
\end{itemize}


\subsection{Juego de la vida}\label{subsec:juego-de-la-vida}
Propuesto por John Horton Conway en 1970 \cite{gardner1970life}, el juego de la vida es un automata celular en dos dimensiones con las siguientes reglas \cite{toffoli1987cellular}:
\begin{itemize}
    \item Se considera 8 vecinos (Vecindad de Moore, r = 1).
    \item Cada celda tiene dos estados posibles “Viva” o “Muerta” ($k=2$).
    \item Las celdas en el estado "Viva", permanecerán en dicho estado en el siguiente paso temporal si tiene 2 o 3 vecinos vivos, de lo contrario morirá.
    \item Las celdas en el estado "Muerta" transicionarán al estado "Viva" solamente si tiene exactamente 3 vecinos en el estado "Viva".
\end{itemize}
