% Preamble
\documentclass[11pt]{article}

% Packages
\usepackage{amsmath}
\usepackage{amssymb}
\usepackage{hyperref}

% Document
\begin{document}

    \tableofcontents


    \section{Introduccion}
    El siguiente trabajo presenta el desarrollo de la implementacion, simulaciones realizadas y analisis de resultados de
    automatas celulares, en particular el juego de la vida y alteraciones del mismo.

    Con fines de producir diversas simulaciones con el objetivo de detectar comportamientos y patrones, se ha realizado
    simulaciones tanto en dos y tres dimensiones, modificando a su vez parametros de entrada.

    \section{Modelo}

    \subsection{Automatas celulares}

    Un automata celular modela un sistema dinamico el cual evoluciona en pasos discretos mediantes reglas o heuristicas,
     las cuales determinan la transicion entre dos estados.

    El sistema se compone de una matriz de celdas que poseen un numero finito de estados discretos y estos se iran actualizando
    de manera sincronica en cada paso temporal.

    Las reglas se caracterizan por ser deterministicas y uniformes en tiempo y espacio. Es decir, que las reglas para la evolucion
    de una celda solamente depende de un vecindario. Donde este vecindario sera definido segun la implementacion y las reglas.

    En una transicion de estados, se define un alcance r, el cual determina el grupo de celdas cercanas que son tenidas
    en cuenta para la regla de transicion. Como es utilizado el valor r varia en funcion de la cantidad de dimensiones de
    la matriz y como se define el vecindario.

    \subsection{Definiciones de vecindad}
    Como se ha mencionado, el alcance es un parametro que describe a un sistema en particualar. Sin embargo, debe tambien
    tenerse en cuenta bajo que definicion de vecindario se trabaja. Se han contemplado los siguietnes.

    En primer lugar el vecindario \textbf{Von Neumann} considera como vecina a aquellas celdas que se encuentran dentro del rango r
    considerando la distancia Manhattan. De forma mas precisa, el conjunto de vecinas se define de la siguiente manera
    $N_{i,j}^{vN}:=\{(k,l) \in L\ |\ |k-i|+ |l - j| \leq r\}$

    Por otro lado, el vecindario \textbf{Moore} considera una celda como vecina si toda las componentes de su posicion se encuentran
    a distancia menor a r. De forma mas precisa el conjunto de celdas vecinas es
    $N_{i,j}^{M}:=\{(k,l) \in L\ |\ |k-i|\leq r\ and\ |l - j| \leq r \}$.

    \subsection{Juego de la vida}
     Propuesto por John Horton Conway en 1970, el juego de la vida es un automata celular en dos dimensiones con las sigueintes reglas
    \begin{itemize}
        \item Se considera 8 vecinos (Vecindad de Moore, r = 1).
        \item Cada celda tiene dos estados posibles “Viva” o “Muerta” ($k=2$).
        \item Las Celdas Vivas, permanecerán vivas en el siguiente paso temporal si tiene 2 o 3 vecinos vivos, de lo contrario morirá.
        \item Las Celdas Muertas se transformarán en Vivas solamente si tiene exactamente 3 vecinos vivos.
    \end{itemize}


    \section{Implementacion}


    \section{Simulaciones}


    \section{Resultados}


    \section{Conclusiones}

    lhh
\end{document}