\section{Conclusiones}\label{sec:conclusiones}

A continuación, se enuncian las conclusiones a las que se han llegado a partir de los resultados obtenidos
en los distintos modelos simulados.

\subsection{Conway en 2D}\label{subsec:conway-en-2d-conc}
A partir del resultado que se ha obtenido en la fig. \ref{fig:conway2d_size_slope_vs_input}, se puede concluir
que la población de celdas vivas disminuye respecto de su población en el paso inicial, independientemente de
la densidad inicial de celdas vivas.
Por lo tanto, Conway en dos dimensiones es un modelo de población decreciente.

Además, si se considera la fig. \ref{fig:conway2d_size_vs_input}, se llega a la conclusión de que, si bien
a medida que aumenta la densidad inicial de celdas vivas, la cantidad de celdas vivas en el paso de equilibrio también,
el decrecimiento se vuelve más pronunciado.
Esto se puede deber a que, ante una mayor densidad de celdas vivas, se vuelve más probable que una celda viva
tenga más de 3 celdas vecinas vivas, lo que la llevaría a morir.

Por último, en base a lo obtenido en las figs. \ref{fig:conway2d_time_vs_input} y \ref{fig:conway2d_distance_slope_vs_input},
se concluye que la rapidez de alejamiento de la celda viva más alejada al centro resulta óptima cuando
$initialLiveCellsProportion = 0.6$.
Aunque se puede argumentar que resulta una coincidencia de la muestra considerada, la baja amplitud en el
desvío estándar otorga una mayor solidez a la afirmación.

\subsection{Cuarentena en 2D}\label{subsec:cuarentena-2D-conc}

Si se necesitan más de 3 vecinos para que una celda reviva, entonces existe una imposiblidad de crecer más allá de la posición más lejana
de alguna celda viva inicial. Esto se debe a que ninguna celda que se encuentre por fuera de la sección inicial podrá contar con más de
3 celdas aledañas vivas.

\subsection{Expansión Circular en 2D}\label{subsec:expansion-circular-2D-conc}

Para tener un crecimiento constante hacia todos lados no es siempre necesario que todas las celdas revivan por cualquier cantidad de vecinos.

\subsection{Expansión Cúbica en 3D}\label{subsec:cubito-3D-conc}

Es posible generar sistemas en los cuales la distancia de la celda más lejana al centro crezca todos los pasos sin tener un aumento en la
cantidad de celdas estrictamente creciente en función del tiempo.

\subsection{Conway en 3D}\label{subsec:conway-en-3D-conc}

A partir de la fig. \ref{fig:conway3d_size_vs_input}, se puede concluir que se alcanza la densidad máxima de
celdas vivas en el equilibrio cuando $initialLiveCellsProportion = 0.5$.

En base a lo obtenido en la fig. \ref{fig:conway3d_size_slope_vs_input}, la expansión de la población
alcanza un máximo cuando la densidad inicial de celdas vivas es del $20\%$.

Si se analiza el tiempo de convergencia al equilibrio, según la fig. \ref{fig:conway3d_time_vs_input},
se puede concluir que el tiempo de convergencia es aproximadamente constante para todas las densidades iniciales,
siendo de 19 pasos temporales.
No así la rapidez de alejamiento de la celda viva más alejada al centro, ya que a partir de
la fig. \ref{fig:conway3d_distance_slope_vs_input} se concluye que la misma aumenta a medida que lo
hace la densidad inicial.

\subsubsection{Conway 2D vs 3D}\label{subsubsec:conway-2D-vs-3D-conc}
Si se comparan los modelos de Conway en dos y tres dimensiones, se puede concluir que el comportamiento
varía significativamente con solo agregar una dimensión.

En primer lugar, analizando las figs. \ref{fig:conway2d_size_slope_vs_input} y \ref{fig:conway3d_size_slope_vs_input},
si bien en ambos modelos la pendiente de crecimiento de la cantidad de celdas vivas
decrementa a medida que aumenta la densidad inicial de celdas vivas,
en el modelo de dos dimensiones la pendiente es negativa, mientras que en el de tres dimensiones es positiva.
Por lo tanto, al agregar una dimensión, el modelo de Conway transiciona de uno que reduce la población de celdas vivas
a uno que la incrementa.

En segundo lugar, si se consideran las figs. \ref{fig:conway2d_time_vs_input} y \ref{fig:conway3d_time_vs_input},
el tiempo de convergencia al equilibrio en el modelo de dos dimensiones varía considerablemente en función
de la densidad inicial de celdas vivas, mientras que en el de tres dimensiones es aproximadamente constante.

Por último, en base a las figs. \ref{fig:conway2d_distance_slope_vs_input} y \ref{fig:conway3d_distance_slope_vs_input},
la rapidez de alejamiento de la celda viva más alejada al centro en el modelo de dos dimensiones se encuentra en el
rango $[0.00, 0.20]$, mientras que en el de tres dimensiones se encuentra en el rango $[0.65, 0.75]$.
Por lo tanto, al agregar una dimensión, este observable incrementa entre tres y cuatro veces.

\subsection{Colapso cúbico}\label{subsec:colapso-cubico-conc}
En base a los resultados obtenidos en la fig. \ref{fig:colapso3d_size_vs_input}, se puede concluir que
la cantidad de celdas vivas en el equilibrio es prácticamente nulo, alcanzando una densidad máxima del
$0.048\%$ cuando $initialLiveCellsProportion = 0.6$.
Además, si se considera la fig. \ref{fig:colapso3d_size_slope_vs_input}, se llega a la conclusión de que
la pendiente de crecimiento de la cantidad de celdas vivas es negativa, independientemente de la densidad inicial,
y decrece considerablemente a medida que aumenta dicha densidad.

Por último, si agregamos la fig. \ref{fig:colapso3d_distance_slope_vs_input} a la discusión, se puede concluir que
el sistema es uno que tiende a colapsar en el centro del dominio inicial, independientemente de la densidad inicial de celdas vivas.