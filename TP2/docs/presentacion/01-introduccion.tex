\section{Introduction}

\begin{frame}{Automatas Celulares}
    \begin{block}{Son arreglos regulares de celdas individuales de la misma clase.}
        Caracteristicas:
        \begin{itemize}
            \item Finitos estados y discretos.
            \item Estados: Actualizan simultáneamente.
            \item Reglas de actualización determinísticas y uniformes.
            \item Solo se considera un vecindario local.
        \end{itemize}
    \end{block}
\end{frame}

\begin{frame}{Fundamentos}
    \begin{block}{Estados:}
        Un estado esta determinado por la celda y el tiempo.
        \\
        $a_{i}^{(t)}$ $\longrightarrow$ Celda $i$ en tiempo $t$

    \end{block}
    \begin{block}{Regla de evolucion:}

        \vspace{1em}

        $a_{i}^{(t)} = f \left[ \sum_{j=-r}^{j=r}  \alpha_{j} a_{i+j}^{(t-1)} \right]$

        \begin{multicols}{3}
            r $\rightarrow$ rango

            $\alpha_j$ $\rightarrow$ constantes enteras

            $f$ $\rightarrow$ regla del automata
        \end{multicols}

    \end{block}
\end{frame}

\begin{frame}{Vecindario}
    \begin{block}{Von Neumann:}
    $N_{i,j}^{vN}:=\{(k,l) \in L\ |\ |k-i|+ |l - j| \leq r\}$
    \end{block}
    \begin{block}{Moore:}
    $N_{i,j}^{M}:=\{(k,l) \in L\ |\ |k-i|\leq r\ \wedge \ |l - j| \leq r \}$
    \end{block}
\end{frame}

